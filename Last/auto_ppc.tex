%-*- coding:UTF-8 -*-
% PPC自动化算法v5.tex
\documentclass[UTF8]{ctexart}
\usepackage{geometry}
\usepackage{enumerate}
\usepackage{amsmath}
\usepackage{amssymb}
\usepackage{dsfont}
\usepackage{amsthm}
\usepackage{listings} %插入代码
\usepackage{xcolor} %代码高亮
\usepackage{blkarray}
\usepackage{diagbox}
\usepackage{tabularx}
\usepackage{graphicx}
\usepackage{caption}
\usepackage{subcaption}
\usepackage{varwidth}
\usepackage{float}
\usepackage{color}
\usepackage{multirow}
\usepackage[all,pdf]{xy}
\usepackage{verbatim}   %comment
\usepackage{cases}
\usepackage{clrscode3e}	% need clrscode3e package which is not included in CTex.
\usepackage{hyperref}
\usepackage{bm}

\geometry{screen}
\hypersetup{
    colorlinks=true,
    bookmarks=true,
    bookmarksopen=false,
    %pdfpagemode=FullScreen,
    pdfstartview=fit,
    pdftitle={DNA-Align},
    pdfauthor={Trasier}
}

% THEOREM Environments --------------------------------------------------------
\newtheorem{thm}{Theorem}[subsection]
\newtheorem{cor}[thm]{Corollary}
\newtheorem{lem}[thm]{Lemma}
\newtheorem{prop}[thm]{Proposition}
\newtheorem{prob}[thm]{Problem}
\newtheorem{mthm}[thm]{Main Theorem}
\theoremstyle{definition}
\newtheorem{defn}[thm]{Definition}
\theoremstyle{remark}
\newtheorem{rem}[thm]{Remark}
\numberwithin{equation}{subsection}
% MATH ------------------------------------------------------------------------
\DeclareMathOperator{\RE}{Re}
\DeclareMathOperator{\IM}{Im}
\DeclareMathOperator{\ess}{ess}
\newcommand{\eps}{\varepsilon}
%\newcommand{\To}{\longrightarrow}  conflict with \package{clrscode3e}p
\newcommand{\h}{\mathcal{H}}
\newcommand{\s}{\mathcal{S}}
\newcommand{\A}{\mathcal{A}}
\newcommand{\J}{\mathcal{J}}
\newcommand{\M}{\mathcal{M}}
\newcommand{\W}{\mathcal{W}}
\newcommand{\X}{\mathcal{X}}
\newcommand{\BOP}{\mathbf{B}}
\newcommand{\BH}{\mathbf{B}(\mathcal{H})}
\newcommand{\KH}{\mathcal{K}(\mathcal{H})}
\newcommand{\Real}{\mathbb{R}}
\newcommand{\Complex}{\mathbb{C}}
\newcommand{\Field}{\mathbb{F}}
\newcommand{\RPlus}{\Real^{+}}
\newcommand{\Polar}{\mathcal{P}_{\s}}
\newcommand{\Poly}{\mathcal{P}(E)}
\newcommand{\EssD}{\mathcal{D}}
\newcommand{\Lom}{\mathcal{L}}
\newcommand{\States}{\mathcal{T}}
\newcommand{\abs}[1]{\left\vert#1\right\vert}
\newcommand{\set}[1]{\left\{#1\right\}}
\newcommand{\seq}[1]{\left<#1\right>}
\newcommand{\norm}[1]{\left\Vert#1\right\Vert}
\newcommand{\essnorm}[1]{\norm{#1}_{\ess}}


% Some setup
\pagestyle{plain}
\geometry{a4paper, top=2cm, bottom=2cm, left=2cm, right=2cm}
\CTEXsetup[format={\raggedright\bfseries\Large}]{section}
\lstset{numbers=left, %设置行号位置
        numberstyle=\small, %设置行号大小
        keywordstyle=\color{blue}, %设置关键字颜色
        commentstyle=\color{purple}, %设置注释颜色
        %frame=single, %设置边框格式
        escapeinside=``, %逃逸字符(1左面的键),用于显示中文
        breaklines, %自动折行
        extendedchars=false, %解决代码跨页时,章节标题,页眉等汉字不显示的问题
        %xleftmargin=2em,xrightmargin=2em, aboveskip=1em, %设置边距
        tabsize=4, %设置tab空格数
        showspaces=false %不显示空格
       }

% About math
\newcommand{\rmnum}[1]{\romannumeral #1}
\newcommand{\Emph}{\textbf}
\newcolumntype{Y}{>{\centering\arraybackslash}X}
\newcommand{\resetcounter}{\setcounter{equation}{0}}
\newcommand{\equsuf}[1][x]{\equiv_{\textit{Suff(#1)}}}	
\newcommand{\Suff}{\textit{Suff}}
\newcommand{\len}[1][x]{\textit{length}_{#1}}

% section deeep to 3 1.1.1
\setcounter{secnumdepth}{3}

\begin{document}

\title{\Huge 流水线自动化的算法设计}
\vspace{2cm}
\author{\Large Trasier}
\date{\today}
\maketitle


\section{Introduction}
\label{sec:intro}
	
	流水线微处理器的设计和实现不能说很困难,但却很复杂。
	设计流水线微处理器的难点之一在于解决冲突,而随着流水线阶段的增加,
	冲突的种类和数量会急剧增加。而人工考虑如何解决冲突需要花费大量的时间和精力,
	仍难免会存在错误。而最终实现的微处理器代码具有大量类似或者相近的逻辑,
	因此,我们不禁产生这样的思考:能不能根据一定的输入,自动生成基于该输入规范的
	流水线微处理器。同时,能不能保证这一实现的正确性。为此,显然需要开发这样一个算法:
	即以流水线微处理器的设计规范作为输入,输出使用Verilog语言描述的流水线微处理器。
	

\section{Notations}
\label{sec:notation}
	
	
	
	使用$Insns$表示指令集合,
	使用$Insns_{MC}$表示Mutli-Cycle类指令集合,
	使用$Insns_{Br}$表示分支类指令。
	
	使用$Stages$表示流水线微处理器的阶段集合,使用$|Stages|$表示流水线微处理器的阶段数目。	\\
	
	使用$Modules$表示流水线微处理器设计的核心模块。
	使用$Port$表示模块的端口,使用$Port_I$表示输入端口,使用$Port_O$表示输出端口。
	

\section{Tool Introduction}
\label{sec:ppc_algov5}

\subsection{Introduction}
	
	前几个版本的算法的核心思想是正向解决问题,即由\Emph{RTL}找到所有可能的指令组合,
	从而指定何时的冲突处理策略。存在的隐患是是否正确处理了冲突对以及多条写指令-单条读指令的情况就一定正确处理了冲突。
	
	v5的算法从另一个角度思考问题。不同的流水线冲突处理的策略都具有相同的目的,
	即正确处理流水线处理器中的所有可能冲突,那么,可能的冲突到底有哪些?
	我们是否能直接构造这些冲突。
	若我们可以构造这些冲突,那么直接制定针对这些冲突的处理策略即可。
	这个过程其实是由仿真测试、发现问题并修改设计的过程。
	我们可以将之理解为:初始时,我们没有设计;然后,我们知道了所有可能的冲突情况,
	我们针对这个情况设计微处理器。因为,我们了解到的是全集的情况,从而我们的策略也一定是完备的。
	
	那么问题变成为如何找到所有的冲突情况,如何通过旁路和阻塞技术最优的处理冲突。

	
\subsection{Framework}

	按照功能及耦合性可以大致分成三个\Emph{Layer}。见图~\ref{fig:v5_architecture}。
	\begin{figure}[H]
	\centering
	\includegraphics[scale=0.6]{v5_arch.png}
    \caption{v5基本框架}
	\label{fig:v5_architecture}
	\end{figure}
	
	\begin{itemize}
	
		\item \Emph{RTL}层
		
		RTL层主要包含算法的输入,其中包括的\Emph{JSON}格式的RTL信息,流水线规范设计,配置文件信息等内容。
		需要具备的基本功能包括对RTL的解析、分类,对规范设计的解析等。
		
		\item \Emph{Algorithm}层
		
		Algorithm层主要包含将RTL信息自动转化为Verilog代码的核心算法,即该层次起承上启下的左右。
		以RTL层作为输入,其输出作为Verilog层的依据。
		其中算法主要针对如下几个方面进行讨论:
		\begin{enumerate}[(1)]
			
			\item 如何自动生成控制信号?
			
			\item 如何自动生成端口MUX?
			
			\item 如何处理基本冲突?
			
			\item 如何处理\Emph{Multi-Cycle}指令引发的冲突?
			
			\item 如何支持中断系统?
			
		\end{enumerate}
		在正确处理上述几个问题后,算法以预先提供的\Emph{核心模块}的RTL作为基本模块,
		自动构建完整的\Emph{控制器}和\Emph{数据通路},最终可得到完整的处理器的Verilog实现。
		
		在Algorithm层,对如下几个方面具备一定的优化能力:
		\begin{enumerate}[(1)]
		
			\item 控制表达式的整合
			
			\item 低利用率MUX的分离
			
			\item 协议的提取
			
		\end{enumerate}
		同时,将围绕如下几个方面进行验证:
		\begin{enumerate}[(1)]
		
			\item 冲突命中的验证
			
			\item 算法正确性的验证
			
			\item 流水线正确性的形式化验证
			
		\end{enumerate}
		
		
		\item \Emph{Verilog}层
		
		在Verilog层主要进行对Verilog代码的仿真和性能进行测试。其中主要功能包括
		冲突覆盖用例生成、使用ISE-QEMU-CrossCompile工具链进行自动仿真测试、解析综合性能仿真报告。
		
	
	\end{itemize}
		
\subsection{Find All Hazard}

	流水线冲突主要包含三类:
	\begin{enumerate}[(1)]
	
		\item 控制冲突
		
		\item 部件冲突
		
		\item 读写冲突
		
	\end{enumerate}
	
	控制冲突主要发生在分支指令时,往往采用固定策略进行处理,如支持延迟槽、分支预测等。
	这里\Emph{PowerPC}体系结构是不支持延迟槽的,分支预测可选,这里选择为不支持。
	实际上,控制冲突的解决策略是固化的,一旦确定了分支类指令及相应的处理策略。
	则控制冲突的处理策略和电路逻辑基本上是固定的。
	
	部件冲突往往采用部件并行解决,当某些情况不支持部件并行时,往往引入\Emph{busy}控制信号解决该类冲突。
	在基于\Emph{PowerPC}体系结构的流水线微处理器的设计中,当多条乘除法指令连续发射时,会引发该类冲突。
	因此,为需要\Emph{Multi-Cycle}完成执行的核心部件增加\Emph{busy}控制信号并增加相应的控制逻辑是有效的处理策略。
	
	上述两类冲突均采用固化的策略进行处理,更主要的是往往采用阻塞辅助解决。
	然而,对于读写冲突,即可以使用旁路转发机制也可以使用阻塞机制进行处理。
	% 可以调研读写冲突的比重
	由于读写操作几乎是任何微处理器最普遍的操作,因此读写冲突在流水线冲突中占据较大比重。
	故正确地处理该类冲突时流水线微处理器的自动化设计的关键。
	
	那么问题转化成流水线微处理器中,究竟有多少读写冲突?
	不放假定当前设计的流水线级数为$n$,ISA的数量为$|S| = c$。
	根据流水线处理器的定义可知,在任何时刻微处理器中都一定存在$n$条指令,这里$nop$指令或者$flushed$指令均算做内。
	因此,假设我们可以构造长度为$n$的排列,并且枚举所有排列中包含冲突的情况。
	那么我们就得到了所有的冲突。显然仅粗略观察至少包含$c^n$种情况(显然不只这些种,考虑寄存器地址)。
	这已经是个近似天文的数字了。那么我们如何快速地去表达流水线冲突。
	
	显然这里可以使用状态压缩,即我并不关心这$n$条指令的汇编具体是什么样的。
	我更关注于我是否覆盖了所有的冲突组合。
	因此,使用3个状态表示指令的状态:0表示不相关;1表示相关的读指令;2表示相关的写指令。
	对于不相关的指令我们毫不关心。因此,所有的冲突情况都可以写成长度为$n$、由$[0, 1 ,2]$组成的序列。
	显然当流水线级数处在一定规模内时,这个序列是可枚举的。
	
	从而我们找到了一种映射方式方式:将所有的冲突情况使用排列表示。
	那么为了还原所有的冲突情况,我们仅仅需要对$[1, 2]$所在位置的指令进行枚举。
	考虑$c \in [100, 500]$这个情况还是太多了,但是万幸的是我们可以对指令类型进行简单的预处理。
	因为很多指令的$RTL$几乎是相同的。经验上看我们可以将指令规模降低为$[10,20]$。
	这样所有的冲突情况一定是可枚举的。
	
	从更加细粒度的角度来看,我们仍然需要考虑几个维度:
	\begin{itemize}
	
		\item 寄存器类型
		
		\item 读指令的读通道
		
		\item 写指令的写通道
		
	\end{itemize}
	这里,寄存器的类型必然是需要枚举的。然而,
	对于\Emph{读指令的读通道}和\Emph{写指令的写通道}我们不必枚举,我们可以假定最坏情况:
	即\Emph{所有}读指令的\Emph{所有}读通道与\Emph{所有}写指令的\Emph{所有}写通道相关。
	若其中部分通道不相关,则一定满足冲突产生条件故不产生转发或阻塞。
	因此,可以看出我们的冲突种类是完备的。
	
	由这些冲突序列,我们可以对是否正确处理\Emph{冲突对}就意味着正确处理任意\Emph{冲突对}组合进行验证。
	同时,也会发现不满足的情况并对原始假设进行修改。

\subsection{Solve Hazard Optimal}
	
	旁路转发和阻塞是解决上述冲突的主要方法。
	通过指令集合的RTL语义可以得到满足转发的最早阶段,将该阶段作为分割点。
	选择旁路转发和阻塞策略。对于每种冲突,探索如何插入旁路转发多路选择器进行转发。
	在必须的阶段插入多路选择器,并判定它可以覆盖的冲突情况。
	因此,原问题转化为精确覆盖问题。
	即使用最少的多路选择器覆盖全部的冲突。
	故可以使用Dancing Links解得最优解。
	
\subsection{Handle Input}

	刚结束的编程之美让我了解到JSON,一直以来的几个版本都使用\Emph{Excel}作为\Emph{RTL}的载体。
	它存在的问题是极度依赖于第三方库或者微软产品。
	这也导致一直用python开发,这是个脚本语言,导致可读性比较差。
	同时,\Emph{RTL}的语法和语义都过于简单。
	使用较复杂的语法和语义可以让\Emph{RTL}更灵活。
	而JSON可以很好的解决这个问题。
	
	
\section{Layer RTL}
\label{sec:layer_rtl}
	
\subsection{RTL-JSON}	

	以文本的格式保存RTL的JSON数据,每个JSON数据间以空行分割。每个JSON其实是一个字典,应该至少包含如下关键字:
	\begin{itemize}
	
		\item \Emph{Insns}表示当前类别的指令集合
		
		\item $\mathbf{P_k}, k \in [1, |Stages|)$表示第$k$阶段的该类别的RTL描述
		
	\end{itemize}
	除上述外,可以采用其他关键字指定同类别共同的RTL及指令集间不同的RTL。
	
	除RTL的JSON信息外,还应当提供处理器的设计规范,同样采用JSON格式。仅包含一个字典,应该至少包含如下几个方面:
	\begin{itemize}
	
		\item \Emph{Modules}表示核心模块
		
		\item \Emph{Stages}表示流水线微处理器的阶段集合
		
		\item \Emph{Insns}表示定制的指令集合且满足$Insns_{Spec} \subseteq Insns_{RTL}$
		
		\item \Emph{Br-Insns}表示分支类指令的集合
		
		\item \Emph{MC-Insns}表示Mutli-Cycle类指令的集合
		
	\end{itemize}
	
	此外,若需要生成得到的微处理器支持中断系统,还应当指令中断系统的JSON格式。
	
	\Emph{RTLParse}即针对RTL的解析,
	JSON的格式设计与解析密不可分,因此将其划分为同一个模块,可以自定义构建JSON格式的解析。
	而\Emph{RTLCollection}针对不同类别的RTL归类并构建模型,其中至少需要包含操作类RTL。
	\Emph{SpecParse}即解析微处理器规范的JSON数据。
	
	在RTL层,需要构建RTLs\_Set用来投放每个阶段的RTL语句,
	同时,需要建立如下模型:
	\begin{itemize}
		
		\item \Emph{pipeline}
		
		\item \Emph{instruction}
		
		\item \Emph{module}
		
	\end{itemize}
	
\section{Layer Algorithm}
\label{sec:layer_algorithm}

	\Emph{Layer Algorithm}的主要任务是实现流水线自动化技术的核心算法,解决与此相关的关键问题。
	该层的算法可以根据微处理器需要支持的关键技术进行修改,其实可以将该层划分为两个子层:
	\begin{itemize}	
		
		\item \Emph{Automation}, 流水线自动化核心算法
		
		\item \Emph{Description}, 使用HDL语言进行描述
	
	\end{itemize}
	
\subsection{Core Module}

\subsubsection{Merge Multiplexer}

	完整的流水线微处理器的数据通路由控制器、核心模块、多路选择器、流水线寄存器及其它必要电路组成。
	多路选择器是其中的重要组成部分,并且数量较多。因此,如何合理并正确的整合多路选择器是流水线自动化技术的难点之一。
	
	从功能上看,可以将多路选择器分为两类:
	\begin{itemize}
	
		\item \Emph{Port Mux}端口多路选择器,即用来从多个核心模块的输入中选择其一
		
		\item \Emph{Bypass Mux}旁路转发多路选择器,即用来从多个旁路转发数据中选择其一
		
	\end{itemize}
	
	这里第一类旁路选择器可以直接通过RTL得到,通过解析RTL,
	我们可以得到对于某个模块的输入端口有多少种输入来源,当输入来源数量超过1时,
	显然该端口前需要放置至少一个多路选择器。
	
	而第二类旁路选择器的获得与冲突处理耦合度较大,因此,需要依赖于冲突处理策略才可以得到。
	
\subsubsection{Hazard}	
	
	这个版本的冲突处理的实现应该相对简单,并且相对清晰。
	
	\begin{mthm}
		对于任何冲突均可以使用阻塞技术进行处理,仅部分冲突可以使用旁路技术处理。
	\end{mthm}
	
	\begin{cor}
	\label{cor:hazard_handle_principle}
		对于任何冲突若不能使用旁路技术进行处理,则一定可以使用阻塞进行处理。
	\end{cor}
	
	\Emph{控制冲突}可以通过由预先定义的分支类指令,构建相应的分支处理策略进行处理,
	这其实是\Emph{half-hard-code},因为分支策略相对固定,这里我们采用不支持延迟槽,
	静态预测分支一定不发生。
	
	\Emph{部件冲突}仅仅发生在多条乘除类指令连续发射时,这类冲突产生的原因在于乘除法运算需要多个周期完成执行。
	因此,若当前乘除队列非空,后续发射了另一条乘除类指令后。由于当前\kw{MDU}被占用,因此存在部件冲突。
	显然,处理该类冲突的基本策略是对\Emph{Multi-Cycle}部件增加\id{Busy}信号,
	当\id{Busy}信号为真时,应当阻塞流水线。
	
	首先,通过测试用例覆盖我们可以得到所有的冲突组合,将其建模为\Emph{Hop}。
	由推论~\ref{cor:hazard_handle_principle}可以将必须使用阻塞处理的排列组合建模为\Emph{StallHazard},
	并且进行整合。同理,将可以使用旁路转发解决的冲突建模为\Emph{BypassHazard},并进行归纳和整合。
	
	由推论~\ref{cor:hazard_handle_principle}反向思考,阻塞的数量已经是最优解,仅仅能够增加不能减少。
	然而,冗余的是\Emph{Bypass Mux}。而这显然是一个精确覆盖问题,即我们需要使用最少的\Emph{Bypass Mux}。
	并仍能够覆盖所有的冲突组合。显然,可以使用\Emph{Dancing Links}处理此问题,关键难点是如何构建\Emph{solution}。
	
\subsubsection{Multi-Cycle}	

	仅仅使用\id{Busy}信号机制即可以支持\Emph{Multi-Cycle}指令,然而存在的问题是性能较低,
	因为乘除类运算指令往往需要十个甚至几十个周期完成执行。
	因此,采用\Emph{“顺序发射,乱序执行”}技术可以一定程度上提高性能。
	
	该技术指指令仍然按照原有顺序进行发射,但是指令的执行顺序与原始顺序不同,但仍然保证不影响程序的相关性。
	这里存在如下几个问题:
	\begin{enumerate}[(1)]
		
		\item MC指令与不相关指令同时完成执行阶段引发部件冲突
		
		这里的部件冲突其实指的是两条指令不可能同时进入流水线寄存器,因此需要新增加阻塞条件,
		即这里的后续指令至少需要阻塞1个周期
		
		\item 如何保存MC指令的信息
		
		这里可以采用影子寄存器技术,可以针对流水线寄存器设置影子寄存器。也可以对单独对指令进行影子寄存,
		并结合分布式译码。技术的难点在于恢复数据信息
		
		\item 如何处理MC指令与后续指令的冲突
		
		显然当MC指令进入影子寄存器后,并不在控制器的指令输入集中,因此需要重新考虑影子寄存器中的指令
		与后续指令间的冲突。
		
	\end{enumerate}
	
	显然,这个技术很靠谱,但是我们需要实际的去考量采用这个技术是不是能提高性能。如何估算?
	可以简单的通过MC指令的频度估算性能的提高

\subsubsection{Interrupt}	

	目前全部做成\Emph{Hard Code},需要进一步对中断机制及中断类型进行抽象。
	
	
\subsubsection{Core Module}
	
\subsubsection{Controller}

\subsubsection{Datapath}

\subsubsection{Expression Merge}

\subsubsection{Mux Seperate}

\subsubsection{Protocol Extract}

\subsubsection{Hit Verification}

\subsubsection{Algorithm Verification}

\subsubsection{Formal Verification}
	
\section{Layer Verilog}
\label{sec:layer_verilog}	

\subsection{Case Generator}

\subsection{Tool Chains}

\subsection{Auto Simulate}p

\subsection{Performance Experiment}
	
\end{document}

	
