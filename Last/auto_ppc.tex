%-*- coding:UTF-8 -*-
% PPC自动化算法v5.tex
\documentclass[hyperref,UTF8]{ctexart}
\usepackage{geometry}
\usepackage{enumerate}
\usepackage{amsmath}
\usepackage{amssymb}
\usepackage{dsfont}
\usepackage{amsthm}
\usepackage{listings} %插入代码
\usepackage{xcolor} %代码高亮
\usepackage{blkarray}
\usepackage{diagbox}
\usepackage{tabularx}
\usepackage{graphicx}
\usepackage{caption}
\usepackage{subcaption}
\usepackage{varwidth}
\usepackage{float}
\usepackage{color}
\usepackage{multirow}
\usepackage[all,pdf]{xy}
\usepackage{verbatim}   %comment
\usepackage{cases}
\usepackage{clrscode3e}	% need clrscode3e package which is not included in CTex.
\usepackage{hyperref}
\usepackage{bm}

\geometry{screen}
\hypersetup{
    colorlinks=true,
    bookmarks=true,
    bookmarksopen=false,
    %pdfpagemode=FullScreen,
    pdfstartview=fit,
    pdftitle={Auto-PPC},
    pdfauthor={Trasier}
}

% THEOREM Environments --------------------------------------------------------
\newtheorem{thm}{Theorem}[subsection]
\newtheorem{cor}[thm]{Corollary}
\newtheorem{lem}[thm]{Lemma}
\newtheorem{prop}[thm]{Proposition}
\newtheorem{prob}[thm]{Problem}
\newtheorem{mthm}[thm]{Main Theorem}
\theoremstyle{definition}
\newtheorem{defn}[thm]{Definition}
\theoremstyle{remark}
\newtheorem{rem}[thm]{Remark}
\numberwithin{equation}{subsection}
% MATH ------------------------------------------------------------------------
\DeclareMathOperator{\RE}{Re}
\DeclareMathOperator{\IM}{Im}
\DeclareMathOperator{\ess}{ess}
\newcommand{\eps}{\varepsilon}
%\newcommand{\To}{\longrightarrow}  conflict with \package{clrscode3e}p
\newcommand{\h}{\mathcal{H}}
\newcommand{\s}{\mathcal{S}}
\newcommand{\A}{\mathcal{A}}
\newcommand{\J}{\mathcal{J}}
\newcommand{\M}{\mathcal{M}}
\newcommand{\W}{\mathcal{W}}
\newcommand{\X}{\mathcal{X}}
\newcommand{\BOP}{\mathbf{B}}
\newcommand{\BH}{\mathbf{B}(\mathcal{H})}
\newcommand{\KH}{\mathcal{K}(\mathcal{H})}
\newcommand{\Real}{\mathbb{R}}
\newcommand{\Complex}{\mathbb{C}}
\newcommand{\Field}{\mathbb{F}}
\newcommand{\RPlus}{\Real^{+}}
\newcommand{\Polar}{\mathcal{P}_{\s}}
\newcommand{\Poly}{\mathcal{P}(E)}
\newcommand{\EssD}{\mathcal{D}}
\newcommand{\Lom}{\mathcal{L}}
\newcommand{\States}{\mathcal{T}}
\newcommand{\abs}[1]{\left\vert#1\right\vert}
\newcommand{\set}[1]{\left\{#1\right\}}
\newcommand{\seq}[1]{\left<#1\right>}
\newcommand{\norm}[1]{\left\Vert#1\right\Vert}
\newcommand{\essnorm}[1]{\norm{#1}_{\ess}}


% Some setup
\pagestyle{plain}
\geometry{a4paper, top=2cm, bottom=2cm, left=2cm, right=2cm}
\CTEXsetup[format={\raggedright\bfseries\Large}]{section}
\lstset{numbers=left, %设置行号位置
        numberstyle=\small, %设置行号大小
        keywordstyle=\color{blue}, %设置关键字颜色
        commentstyle=\color{purple}, %设置注释颜色
        %frame=single, %设置边框格式
        escapeinside=``, %逃逸字符(1左面的键),用于显示中文
        breaklines, %自动折行
        extendedchars=false, %解决代码跨页时,章节标题,页眉等汉字不显示的问题
        %xleftmargin=2em,xrightmargin=2em, aboveskip=1em, %设置边距
        tabsize=4, %设置tab空格数
        showspaces=false %不显示空格
       }

% About math
\newcommand{\rmnum}[1]{\romannumeral #1}
\newcommand{\Emph}{\textbf}
\newcolumntype{Y}{>{\centering\arraybackslash}X}
\newcommand{\resetcounter}{\setcounter{equation}{0}}
\newcommand{\equsuf}[1][x]{\equiv_{\textit{Suff(#1)}}}	
\newcommand{\Suff}{\textit{Suff}}
\newcommand{\len}[1][x]{\textit{length}_{#1}}

% section deeep to 3 1.1.1
\setcounter{secnumdepth}{3}

\begin{document}

\title{\Huge 流水线自动化的算法设计}
\vspace{2cm}
\author{\Large Trasier}
\date{\today}
\maketitle


\section{Introduction}
\label{sec:intro}
	
	流水线微处理器的设计和实现不能说很困难,但却很复杂。
	设计流水线微处理器的难点之一在于解决冲突,而随着流水线阶段的增加,
	冲突的种类和数量会急剧增加。而人工考虑如何解决冲突需要花费大量的时间和精力,
	仍难免会存在错误。而最终实现的微处理器代码具有大量类似或者相近的逻辑,
	因此,我们不禁产生这样的思考:能不能根据一定的输入,自动生成基于该输入规范的
	流水线微处理器。同时,能不能保证这一实现的正确性。为此,显然需要开发这样一个算法:
	即以流水线微处理器的设计规范作为输入,输出使用Verilog语言描述的流水线微处理器。
	

\section{Notations}
\label{sec:notation}
	
	
	
	使用$Insns$表示指令集合,
	使用$Insns_{MC}$表示Mutli-Cycle类指令集合,
	使用$Insns_{Br}$表示分支类指令。
	
	使用$Stages$表示流水线微处理器的阶段集合,使用$|Stages|$表示流水线微处理器的阶段数目。	\\
	
	使用$Modules$表示流水线微处理器设计的核心模块。
	使用$Port$表示模块的端口,使用$Port_I$表示输入端口,使用$Port_O$表示输出端口。
	

\section{Tool Introduction}
\label{sec:ppc_algov5}

\subsection{Introduction}
	
	前几个版本的算法的核心思想是正向解决问题,即由\Emph{RTL}找到所有可能的指令组合,
	从而指定何时的冲突处理策略。存在的隐患是是否正确处理了冲突对以及多条写指令-单条读指令的情况就一定正确处理了冲突。
	
	v5的算法从另一个角度思考问题。不同的流水线冲突处理的策略都具有相同的目的,
	即正确处理流水线处理器中的所有可能冲突,那么,可能的冲突到底有哪些?
	我们是否能直接构造这些冲突。
	若我们可以构造这些冲突,那么直接制定针对这些冲突的处理策略即可。
	这个过程其实是由仿真测试、发现问题并修改设计的过程。
	我们可以将之理解为:初始时,我们没有设计;然后,我们知道了所有可能的冲突情况,
	我们针对这个情况设计微处理器。因为,我们了解到的是全集的情况,从而我们的策略也一定是完备的。
	
	那么问题变成为如何找到所有的冲突情况,如何通过旁路和阻塞技术最优的处理冲突。

	
\subsection{Framework}

	按照功能及耦合性可以大致分成三个\Emph{Layer}。见图~\ref{fig:v5_architecture}。
	\begin{figure}[H]
	\centering
	\includegraphics[scale=0.6]{v5_arch.png}
    \caption{v5基本框架}
	\label{fig:v5_architecture}
	\end{figure}
	
	\begin{itemize}
	
		\item \Emph{RTL}层
		
		RTL层主要包含算法的输入,其中包括的\Emph{JSON}格式的RTL信息,流水线规范设计,配置文件信息等内容。
		需要具备的基本功能包括对RTL的解析、分类,对规范设计的解析等。
		
		\item \Emph{Algorithm}层
		
		Algorithm层主要包含将RTL信息自动转化为Verilog代码的核心算法,即该层次起承上启下的左右。
		以RTL层作为输入,其输出作为Verilog层的依据。
		其中算法主要针对如下几个方面进行讨论:
		\begin{enumerate}[(1)]
			
			\item 如何自动生成控制信号?
			
			\item 如何自动生成端口MUX?
			
			\item 如何处理基本冲突?
			
			\item 如何处理\Emph{Multi-Cycle}指令引发的冲突?
			
			\item 如何支持中断系统?
			
		\end{enumerate}
		在正确处理上述几个问题后,算法以预先提供的\Emph{核心模块}的RTL作为基本模块,
		自动构建完整的\Emph{控制器}和\Emph{数据通路},最终可得到完整的处理器的Verilog实现。
		
		在Algorithm层,对如下几个方面具备一定的优化能力:
		\begin{enumerate}[(1)]
		
			\item 控制表达式的整合
			
			\item 低利用率MUX的分离
			
			\item 协议的提取
			
		\end{enumerate}
		同时,将围绕如下几个方面进行验证:
		\begin{enumerate}[(1)]
		
			\item 冲突命中的验证
			
			\item 算法正确性的验证
			
			\item 流水线正确性的形式化验证
			
		\end{enumerate}
		
		
		\item \Emph{Verilog}层
		
		在Verilog层主要进行对Verilog代码的仿真和性能进行测试。其中主要功能包括
		冲突覆盖用例生成、使用ISE-QEMU-CrossCompile工具链进行自动仿真测试、解析综合性能仿真报告。
		
	
	\end{itemize}
		
\subsection{Find All Hazard}

	流水线冲突主要包含三类:
	\begin{enumerate}[(1)]
	
		\item 控制冲突
		
		\item 部件冲突
		
		\item 读写冲突
		
	\end{enumerate}
	
	控制冲突主要发生在分支指令时,往往采用固定策略进行处理,如支持延迟槽、分支预测等。
	这里\Emph{PowerPC}体系结构是不支持延迟槽的,分支预测可选,这里选择为不支持。
	实际上,控制冲突的解决策略是固化的,一旦确定了分支类指令及相应的处理策略。
	则控制冲突的处理策略和电路逻辑基本上是固定的。
	
	部件冲突往往采用部件并行解决,当某些情况不支持部件并行时,往往引入\Emph{busy}控制信号解决该类冲突。
	在基于\Emph{PowerPC}体系结构的流水线微处理器的设计中,当多条乘除法指令连续发射时,会引发该类冲突。
	因此,为需要\Emph{Multi-Cycle}完成执行的核心部件增加\Emph{busy}控制信号并增加相应的控制逻辑是有效的处理策略。
	
	上述两类冲突均采用固化的策略进行处理,更主要的是往往采用阻塞辅助解决。
	然而,对于读写冲突,即可以使用旁路转发机制也可以使用阻塞机制进行处理。
	% 可以调研读写冲突的比重
	由于读写操作几乎是任何微处理器最普遍的操作,因此读写冲突在流水线冲突中占据较大比重。
	故正确地处理该类冲突时流水线微处理器的自动化设计的关键。
	
	那么问题转化成流水线微处理器中,究竟有多少读写冲突?
	不放假定当前设计的流水线级数为$n$,ISA的数量为$|S| = c$。
	根据流水线处理器的定义可知,在任何时刻微处理器中都一定存在$n$条指令,这里$nop$指令或者$flushed$指令均算做内。
	因此,假设我们可以构造长度为$n$的排列,并且枚举所有排列中包含冲突的情况。
	那么我们就得到了所有的冲突。显然仅粗略观察至少包含$c^n$种情况(显然不只这些种,考虑寄存器地址)。
	这已经是个近似天文的数字了。那么我们如何快速地去表达流水线冲突。
	
	显然这里可以使用状态压缩,即我并不关心这$n$条指令的汇编具体是什么样的。
	我更关注于我是否覆盖了所有的冲突组合。
	因此,使用3个状态表示指令的状态:0表示不相关;1表示相关的读指令;2表示相关的写指令。
	对于不相关的指令我们毫不关心。因此,所有的冲突情况都可以写成长度为$n$、由$[0, 1 ,2]$组成的序列。
	显然当流水线级数处在一定规模内时,这个序列是可枚举的。
	
	从而我们找到了一种映射方式方式:将所有的冲突情况使用排列表示。
	那么为了还原所有的冲突情况,我们仅仅需要对$[1, 2]$所在位置的指令进行枚举。
	考虑$c \in [100, 500]$这个情况还是太多了,但是万幸的是我们可以对指令类型进行简单的预处理。
	因为很多指令的$RTL$几乎是相同的。经验上看我们可以将指令规模降低为$[10,20]$。
	这样所有的冲突情况一定是可枚举的。
	
	从更加细粒度的角度来看,我们仍然需要考虑几个维度:
	\begin{itemize}
	
		\item 寄存器类型
		
		\item 读指令的读通道
		
		\item 写指令的写通道
		
	\end{itemize}
	这里,寄存器的类型必然是需要枚举的。然而,
	对于\Emph{读指令的读通道}和\Emph{写指令的写通道}我们不必枚举,我们可以假定最坏情况:
	即\Emph{所有}读指令的\Emph{所有}读通道与\Emph{所有}写指令的\Emph{所有}写通道相关。
	若其中部分通道不相关,则一定满足冲突产生条件故不产生转发或阻塞。
	因此,可以看出我们的冲突种类是完备的。
	
	由这些冲突序列,我们可以对是否正确处理\Emph{冲突对}就意味着正确处理任意\Emph{冲突对}组合进行验证。
	同时,也会发现不满足的情况并对原始假设进行修改。

\subsection{Solve Hazard Optimal}
	
	旁路转发和阻塞是解决上述冲突的主要方法。
	通过指令集合的RTL语义可以得到满足转发的最早阶段,将该阶段作为分割点。
	选择旁路转发和阻塞策略。对于每种冲突,探索如何插入旁路转发多路选择器进行转发。
	在必须的阶段插入多路选择器,并判定它可以覆盖的冲突情况。
	因此,原问题转化为精确覆盖问题。
	即使用最少的多路选择器覆盖全部的冲突。
	故可以使用Dancing Links解得最优解。
	
\subsection{Handle Input}

	刚结束的编程之美让我了解到JSON,一直以来的几个版本都使用\Emph{Excel}作为\Emph{RTL}的载体。
	它存在的问题是极度依赖于第三方库或者微软产品。
	这也导致一直用python开发,这是个脚本语言,导致可读性比较差。
	同时,\Emph{RTL}的语法和语义都过于简单。
	使用较复杂的语法和语义可以让\Emph{RTL}更灵活。
	而JSON可以很好的解决这个问题。
	
	
\section{Layer RTL}
\label{sec:layer_rtl}
	
\subsection{RTL-JSON}	

	以文本的格式保存RTL的JSON数据,每个JSON数据间以空行分割。每个JSON其实是一个字典,应该至少包含如下关键字:
	\begin{itemize}
	
		\item \Emph{Insns}表示当前类别的指令集合
		
		\item $\mathbf{P_k}, k \in [1, |Stages|)$表示第$k$阶段的该类别的RTL描述
		
	\end{itemize}
	除上述外,可以采用其他关键字指定同类别共同的RTL及指令集间不同的RTL。
	
	除RTL的JSON信息外,还应当提供处理器的设计规范,同样采用JSON格式。仅包含一个字典,应该至少包含如下几个方面:
	\begin{itemize}
	
		\item \Emph{Modules}表示核心模块
		
		\item \Emph{Stages}表示流水线微处理器的阶段集合
		
		\item \Emph{Insns}表示定制的指令集合且满足$Insns_{Spec} \subseteq Insns_{RTL}$
		
		\item \Emph{Br-Insns}表示分支类指令的集合
		
		\item \Emph{MC-Insns}表示Mutli-Cycle类指令的集合
		
	\end{itemize}
	
	此外,若需要生成得到的微处理器支持中断系统,还应当指令中断系统的JSON格式。
	
	\Emph{RTLParse}即针对RTL的解析,
	JSON的格式设计与解析密不可分,因此将其划分为同一个模块,可以自定义构建JSON格式的解析。
	而\Emph{RTLCollection}针对不同类别的RTL归类并构建模型,其中至少需要包含操作类RTL。
	\Emph{SpecParse}即解析微处理器规范的JSON数据。
	
	在RTL层,需要构建RTLs\_Set用来投放每个阶段的RTL语句,
	同时,需要建立如下模型:
	\begin{itemize}
		
		\item \Emph{pipeline}
		
		\item \Emph{instruction}
		
		\item \Emph{module}
		
	\end{itemize}
	
\subsection{Interface}
	
	\begin{table}[H]
    \centering
	\caption{Layer\-RTL相关}
	\begin{tabular}{|c|c|c|}
    \hline
		名称	&	描述	&	返回值		\\
    \hline
		\proc{dumpRtl}	& 返回每条指令的RTL	&	\{'add':[[rtls@P0], [rtls@P1], $\cdots$ [rtls@Pk]]\}	\\
		\proc{dumpSpec}	& 返回微处理器的specification	& \{'stage':[stageNames], $\cdots$\}	\\
    \hline
	\end{tabular}
	\end{table}
	
\subsection{Task}

	\begin{enumerate}[(1)]
	
		\item 参考\href{https://github.com/Turf1013/PPC_AutoTool/blob/zyx_testBin/Last/json.format}{json.format}的格式,
		定制RTL及微处理器Specification的格式
		
		\item 参考EXCEL设计较为复杂的文法,可以参考CMU论文的附录
		
	\end{enumerate}
	
\section{Layer Algorithm}
\label{sec:layer_algorithm}

	\Emph{Layer Algorithm}的主要任务是实现流水线自动化技术的核心算法,解决与此相关的关键问题。
	该层的算法可以根据微处理器需要支持的关键技术进行修改,其实可以将该层划分为两个子层:
	\begin{itemize}	
		
		\item \Emph{Automation}, 流水线自动化核心算法
		
		\item \Emph{Description}, 使用HDL语言进行描述
	
	\end{itemize}
	
\subsection{Core Function}

\subsubsection{Merge Multiplexer}

	完整的流水线微处理器的数据通路由控制器、核心模块、多路选择器、流水线寄存器及其它必要电路组成。
	多路选择器是其中的重要组成部分,并且数量较多。因此,如何合理并正确的整合多路选择器是流水线自动化技术的难点之一。
	
	从功能上看,可以将多路选择器分为两类:
	\begin{itemize}
	
		\item \Emph{Port Mux}端口多路选择器,即用来从多个核心模块的输入中选择其一
		
		\item \Emph{Bypass Mux}旁路转发多路选择器,即用来从多个旁路转发数据中选择其一
		
	\end{itemize}
	
	这里第一类旁路选择器可以直接通过RTL得到,通过解析RTL,
	我们可以得到对于某个模块的输入端口有多少种输入来源,当输入来源数量超过1时,
	显然该端口前需要放置至少一个多路选择器。
	
	而第二类旁路选择器的获得与冲突处理耦合度较大,因此,需要依赖于冲突处理策略才可以得到。
	
\subsubsection{Hazard}	
	
	这个版本的冲突处理的实现应该相对简单,并且相对清晰。
	
	\begin{mthm}
		对于任何冲突均可以使用阻塞技术进行处理,仅部分冲突可以使用旁路技术处理。
	\end{mthm}
	
	\begin{cor}
	\label{cor:hazard_handle_principle}
		对于任何冲突若不能使用旁路技术进行处理,则一定可以使用阻塞进行处理。
	\end{cor}
	
	\Emph{控制冲突}可以通过由预先定义的分支类指令,构建相应的分支处理策略进行处理,
	这其实是\Emph{half-hard-code},因为分支策略相对固定,这里我们采用不支持延迟槽,
	静态预测分支一定不发生。
	
	\Emph{部件冲突}仅仅发生在多条乘除类指令连续发射时,这类冲突产生的原因在于乘除法运算需要多个周期完成执行。
	因此,若当前乘除队列非空,后续发射了另一条乘除类指令后。由于当前\kw{MDU}被占用,因此存在部件冲突。
	显然,处理该类冲突的基本策略是对\Emph{Multi-Cycle}部件增加\id{Busy}信号,
	当\id{Busy}信号为真时,应当阻塞流水线。
	
	首先,通过测试用例覆盖我们可以得到所有的冲突组合,将其建模为\Emph{Hop}。
	由推论~\ref{cor:hazard_handle_principle}可以将必须使用阻塞处理的排列组合建模为\Emph{StallHazard},
	并且进行整合。同理,将可以使用旁路转发解决的冲突建模为\Emph{BypassHazard},并进行归纳和整合。
	
	由推论~\ref{cor:hazard_handle_principle}反向思考,阻塞的数量已经是最优解,仅仅能够增加不能减少。
	然而,冗余的是\Emph{Bypass Mux}。而这显然是一个精确覆盖问题,即我们需要使用最少的\Emph{Bypass Mux}。
	并仍能够覆盖所有的冲突组合。显然,可以使用\Emph{Dancing Links}处理此问题,关键难点是如何构建\Emph{solution}。
	
\subsubsection{Multi-Cycle}	

	仅仅使用\id{Busy}信号机制即可以支持\Emph{Multi-Cycle}指令,然而存在的问题是性能较低,
	因为乘除类运算指令往往需要十个甚至几十个周期完成执行。
	因此,采用\Emph{“顺序发射,乱序执行”}技术可以一定程度上提高性能。
	
	该技术指指令仍然按照原有顺序进行发射,但是指令的执行顺序与原始顺序不同,但仍然保证不影响程序的相关性。
	这里存在如下几个问题:
	\begin{enumerate}[(1)]
		
		\item MC指令与不相关指令同时完成执行阶段引发部件冲突
		
		这里的部件冲突其实指的是两条指令不可能同时进入流水线寄存器,因此需要新增加阻塞条件,
		即这里的后续指令至少需要阻塞1个周期
		
		\item 如何保存MC指令的信息
		
		这里可以采用影子寄存器技术,可以针对流水线寄存器设置影子寄存器。也可以对单独对指令进行影子寄存,
		并结合分布式译码。技术的难点在于恢复数据信息
		
		\item 如何处理MC指令与后续指令的冲突
		
		显然当MC指令进入影子寄存器后,并不在控制器的指令输入集中,因此需要重新考虑影子寄存器中的指令
		与后续指令间的冲突。
		
	\end{enumerate}
	
	显然,这个技术很靠谱,但是我们需要实际的去考量采用这个技术是不是能提高性能。如何估算?
	可以简单的通过MC指令的频度估算性能的提高

\subsubsection{Interrupt}	

	目前全部做成\Emph{Hard Code},需要进一步对中断机制及中断类型进行抽象。
	
	
\subsubsection{Core Module}
	
	核心模块主要由如下几个部分组成
	\begin{itemize}	
		\item \kw{PC}, \kw{NPC}
		\item \kw{GPR}, \kw{SPR}
		\item \kw{CR}, \kw{CR\_ALU}
		\item \kw{ALU}, \kw{MDU}
		\item \kw{IM}, \kw{DM}
		\item \kw{Converter}
	\end{itemize}
	
\subsubsection{Controller}

	控制器的主要逻辑包括
	\begin{enumerate}[(1)]
		
		\item 写信号的控制逻辑
		
		\item 核心模块操作信号的控制逻辑
		
		\item Port\ Mux的片选信号
		
		\item Bypass\ Mux的片选信号
		
		\item 阻塞信号的控制逻辑
		
		\item Flush信号的控制逻辑
		
		\item 中断系统相关信号
		
	\end{enumerate}

\subsubsection{Datapath}

	数据通路的主要逻辑是实例化相关核心模块、控制器,并在需要位置插入多路选择器。
	除此外,数据通路还应该实例化流水线寄存器。并潜在的包含部分的额外逻辑。数据通路的生成完全依赖于
	其他模块的完成。

\subsubsection{Expression Merge}
	
	\begin{defn}
	\label{defn:expression_merge}
	\Emph{Expression Merge} 指的是将整个工程的电路逻辑表达式进行分析,并经过一定合理的分析选择一定频度的表达式进行合并。
	再保持原有时钟频率规模的前提下实现资源的降低。
	\end{defn}
	
	做\Emph{Expression Merge}的契机主要是自动工具生成的流水线微处理器,远远大于同等指令规模的MIPS微处理器。
	因此,我们不禁思考生成的微处理器的资源规模是否和自动生成verilog代码的策略有关。后续实验会围绕这个问题进行讨论。
	对于生成的微处理器,尤其是控制器模块,包含大量的组合逻辑。因此,并且由于冲突条件的存在。导致很多子电路出现的
	频度很高。因此,通过统计这全局控制逻辑的表达式,并对高频的表达式进行合并,从而将并行电路转化为级联电路。
	显然,这样的策略可以实现减少资源的目的,但是过度的进行\Emph{Expression Merge}一定会较大程度上降低主频,这显然是个负效应。
	因此,我们需要选择合理的标准来执行这个操作。
	

\subsubsection{Mux Seperate}

	\begin{defn}
	\label{defn:mux_seperate}
	\Emph{Mux Seperate} 指的是将一个较多端口未使用的多路选择器转换为几个多路选择器的级联,一定程度上减少了所需资源。
	\end{defn}
	
	做\Emph{Mux Seperate}的契机主要是随着流水线阶段的增加、指令规模的增加,\Emph{Datapath}中会产生越来越多的Mux。
	然而,并不是每个Mux的利用率都是100\%。因为很多Mux的数据端口并不需要被使用,而多路选择器会随着端口的增加急剧增加。
	因此,将较多端口未使用的Mux拆分成几个小规模的多路选择器的级联可以在保持原有逻辑正确的前提下,较大规模降低资源使用。
	同时,保持原有的主频。

\subsubsection{Protocol Extract}

	\href{https://users.ece.cmu.edu/~jhoe/distribution/2010/nurvitadhi.pdf}{CMU}的论文很好的贡献是对
	冲突处理协议的总结。这个协议其实就是实现\Emph{旁路}和\Emph{阻塞}机制需要哪些控制信号,并且这些控制
	信号的功能都是什么,逻辑由哪些控制。为了支持\Emph{Cache}和\Emph{MMU}是否需要对协议进行修改。

\subsubsection{Hit Verification}

	处理流水线冲突的技术有很多种,你的设计中可以混合使用多种技术。但是,这并不能保证一个核心问题:
	你考虑了所有可能的冲突以及你处理的所有的冲突。那么我们能不能再设计阶段就对冲突进行模拟和验证。
	我们将这样的验证成为\Emph{Hit Verification}。
	
	\begin{defn}
	\label{defn:hit_verify}
	\Emph{Hit Verification}指在设计阶段对冲突处理策略是否可以覆盖所有可能的冲突组合进行验证,验证的主要
	目的是为了检测冲突处理的完备性。
	\end{defn}
	
	显然,正确的流水线微处理器的Hit Verification一定是正确的。我认为Hit Verification应该主要包含如下两个部分:
	\begin{itemize}
		\item 输入的文法设计
		\item 冲突组合序列生成
		\item 冲突覆盖性检测
	\end{itemize}
	\href{https://github.com/Turf1013/PPC_AutoTool/tree/master/IG}{IG}是目前实现的冲突序列生成的工具。
	理论上生成所有指令的组合更具备全面性,但是由于这里仅仅是对覆盖完备性的验证。并不是对指令功能的验证。
	因此,只要能够验证完备性即可。这里可以提取参数\Emph{覆盖率}用来衡量设计的优劣。

\subsubsection{Algorithm Verification}

	算法的验证主要包含如下几个方面:
	\begin{enumerate}[(1)]
		\item 算法确实可以解决流水线冲突的覆盖问题
		\item 得到的解决策略确实是最优的
	\end{enumerate}

\subsubsection{Formal Verification}

	基本思路是模型检测。
	

\subsection{Cover算法}	
\label{subsec:algo_cover}

\subsubsection{Introduction}
	
	Cover算法首先对\Emph{读写冲突}的状态进行抽象得到集合\id{Q},因此对于任意\id{n}阶流水线的所有指令状态的排列总数
	\id{P_{hazard}},一定有
	\[
		|P_{hazard}| \le |Q|^{n}.
	\]
	显然,可以使用\Emph{预处理全排列}并\Emph{剪枝}的方法找到所有\Emph{有效}的冲突序列。这样的集合一定是一个完备的集合。
	证明依赖于\Emph{全排列的完备性}。
	
	在此基础上,Cover算法旨在解决如下几个问题:
	\begin{enumerate}[(1)]
	
		\item 生成完备的冲突序列
		
		\item 针对特定的冲突序列,生成处理策略
		
		\item 选择若干种最优策略组合,实现完备的冲突处理策略
		
	\end{enumerate}
	
\subsubsection{完备的冲突序列}
	
	不妨定义$p = [x_1, \cdots, x_n]$表示有效的冲突序列,其中$x_1$表示处在最后一个阶段的指令状态,$x_n$表示刚刚进入流水线的指令状态。
	
	\begin{defn}
	\label{defn:valid_perm}
	有效的冲突序列指
	\end{defn}
	
	令\id{rStg}表示读寄存器最早的阶段,
	不妨令$rIdx$表示$x_{rIdx} = Q_r, \text{ 并且 } \forall i < rIdx, x_i \neq Q_r$,相似地,令
		  $wIdx$表示$x_{wIdx} = Q_w, \text{ 并且 } \forall i < wIdx, x_i \neq Q_w$。
	
	\begin{thm}
	\label{thm:valid_perm_property}
	任意有效的冲突序列必定满足如下性质:
	\begin{enumerate}[(1)]
		
		\item
		\begin{equation}
			\forall i, x_i \notin \{Q_r, Q_w\}, i \le n - rStg		\label{equ:cover_begin}
		\end{equation}
		
		\item
		\begin{equation}
			\exists i, j, x_i = Q_r, x_j = Q_w						\label{equ:cover_exists}
		\end{equation}
		
		\item
		\begin{equation}
			\forall i, x_i = Q_w, \exists j, j > i \text{ and } x_j = Q_r	\label{equ:cover_rwcomp}
		\end{equation}
		
	\end{enumerate}
	\end{thm}
	
	\begin{proof}
	\label{proof:valid_perm_property}
		
		性质~\ref{equ:cover_begin}是显然的,由\Emph{流水线}设计的一般思路。$rStg$阶段前的指令一定不包含读操作和写操作。
		而该序列是逆序的,因此任何$n-rStg$阶段后的状态一定不含$Q_r$和$Q_w$。
		
		性质~\ref{equ:cover_exists}也同样简单可证,由\Emph{数据冒险}定义可知在有效的序列中一定同时含有$Q_r$和$Q_w$。
		
		不妨令$P$表示经过上述两次过滤余下的排列,不妨令$P'$表示满足性质~\ref{equ:cover_rwcomp}的排列的集合,将分为如下两个部分进行证明:
		\begin{itemize}
			\item $\forall p \in P'$, \id{p}是无效的
			
			\item $P \setminus P' \supseteq P_{hazard}$
		\end{itemize}
		根据定义可知有$\exists i, x_i = Q_w, \forall j>i, x_j \neq Q_r$。这意味着对于处在$i$所在阶段的前驱指令,没有后继指令。
		显然,这条指令并不在冲突集合中,因此是无效的。
		
		而正因为这条指令并不在冲突集合中,那么将所有满足~\ref{equ:cover_exists}的$x_i$全部替换为$Q_n$,并不影响整体的完备性。
		而这样一个排列$p \notin P'$,那么$p \in P \setminus P'$
		
	\end{proof}
	
	
	% 这里注意排列的定义,这里使用排列是不准确的。
	使用\Emph{DFS}可以得到对于状态集合\id{Q}的任意可重复排列,使用如下剪枝策略可得到集合$P$并且有$P \supseteq P_{hazard}$.
	\begin{enumerate}
	
		\item 根据~\ref{equ:cover_begin}过滤任何\Emph{读写状态}在$n-rStg$后面阶段的序列。
		
		\item 根据~\ref{equ:cover_exists}过滤任何不同时包含\Emph{读写状态}的序列。
		
		\item 根据~\ref{equ:cover_rwcomp}过滤任何\Emph{写状态}后没有\Emph{读状态}的序列。
		
	\end{enumerate}
	
	
	
	
\section{Layer Verilog}
\label{sec:layer_verilog}	

\subsection{Case Generator}

	覆盖所有可能冲突的序列是验证处理器正确性的基础,同时也是验证阶段的关键之一。任何无法通过完备测试用例的设计,
	不足以进行后续阶段的测试。\href{https://github.com/Turf1013/PPC_AutoTool/tree/master/IG}{Case Generator}已经
	实现了所需要的基本功能。理论上可以生成所有指令的冲突组合,同时也支持指令集的定义。

\subsection{Tool Chains}
	
	新的工具链主要包括如下工具:
	\begin{itemize}
		\item \Emph{Last} 自动化工具,自动生成Verilog代码
		\item \Emph{Hit} 覆盖检测工具,检测处理器的设计能否覆盖潜在的所有冲突
		\item \Emph{IG} 测试用例生成工具,生成汇编级的测试用例
		\item \Emph{AS} 自动仿真工具,使用Tcl自动进行仿真,同时对比Qemu的log进行验证
		\item \Emph{PE} 将通过上述仿真测试的工程丢到性能实验中进行实验,并提取实验报告
	\end{itemize}
	
\subsection{Auto Simulate}

	\href{https://github.com/Turf1013/PPC_AutoTool/tree/master/PPC}{PowerPC交叉开发环境}是使用C语言进行交叉开发验证的环境,
	其中
	\href{https://github.com/Turf1013/PPC_AutoTool/blob/master/PPC/global/powerpc.lds}{powerpc.lds}是数据段、代码段等的基地址;
	
	\href{https://github.com/Turf1013/PPC_AutoTool/blob/master/PPC/global/Makefile.global}{Makefile.global}是全局的Makefile文件,
	包含所需要的库函数,如果需要增加库函数,同时需要修改该文件;
	
	\href{https://github.com/Turf1013/PPC_AutoTool/blob/master/PPC/global/int_entry.S}{int\_entry.S}是PowerPC的中断处理框架,
	其中\proc{handle\_hw\_int}需要在C程序中实现。
	
	预先提供部分C程序作为测试集,目前拟采用算法为主。数据随机生成即可,因此库函数还需要实现这样一个伪随机数。

\subsection{Performance Experiment}
	
\end{document}

	
